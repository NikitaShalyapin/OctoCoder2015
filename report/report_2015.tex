% This package designed and commented in russian koi8-r encoding.
%
% Класс документов по ГОСТ 7.32-2001 "Отчёт о научно-исследовательской работе"
% на основе ГОСТ 2.105-95
% Автор - Алексей Томин, с помощью списка рассылки latex-gost-request@ice.ru,
%  "extreport.cls", "lastpage.sty" и конференции RU.TEX
% Лицензия GPL
% Все вопросы, замечания кки пожелания сюда: mailto:alxt@yandex.ru
% Дальнейшая разработка и поддержка - Михаил Конник,
% связаться можно по адресу mydebianblog@gmail.com

\documentclass[utf8,usehyperref,12pt]{G7-32}
\usepackage[T2A]{fontenc}
\usepackage[utf8]{inputenc} %% ваша любимая кодировка здесь
\usepackage[english,russian]{babel} %% это необходимо для включения переносов
\usepackage{float}
\usepackage[pdftex]{graphicx}
\graphicspath{{pictures/}}
\usepackage{float}
\usepackage{listings}
% TODOs
\usepackage[%
  colorinlistoftodos,
  shadow
]{todonotes}

\graphicspath{{pictures/}}

\TableInChaper % таблицы будут нумероваться в пределах раздела
\PicInChaper   % рисунки будут нумероваться в пределах раздела
\setlength\GostItemGap{2mm}% для красоты можно менять от 0мм

% Определяем заголовки для титульной страницы
\NirOrgLongName{\textsc{Федеральное государственное бюджетное образовательное учреждение \linebreak высшего профессионального образования \linebreak Санкт-Петербургский государственный политехнический университет }} %% Полное название организации

\NirBoss{Проректор по научной работе}{Д.Ю. Райчук} %% Заказчик, утверждающий НИР
\NirManager{Доцент, к.т.н.}{Н.~В.~Богач} %% Название организации

\NirYear{2015}%% если нужно поменять год отчёта; если закомментировано, ставится текущий год
\NirTown{Санкт-Петербург,} %% город, в котором написан отчёт
% по проекту \No8550: 

% \NirIsAnnotacion{АННОТАЦИОННЫЙ } %% Раскомментируйте, если это аннотационный отчёт

\NirUdk{УДК \No }
\NirGosNo{Регистрационный \No 123123123}

\NirStage{Этап \No 1}{промежуточный}{} %%% Этап НИР: {номер этапа}{вид отчёта - промежуточный или заключительный}{название этапа}

\bibliographystyle{unsrt} %Стиль библиографических ссылок БибТеХа

%%%%%%%<------------- НАЧАЛО ДОКУМЕНТА
\begin{document}

\usefont{T2A}{ftm}{m}{} %%% Использование шрифтов Т2 для возможности скопировать текст из PDF-файлов.

\frontmatter %%% <-- это выключает нумерацию ВСЕГО; здесь начинаются ненумерованные главы типа Исполнители, Обозначения и прочее

\NirTitle{\textbf{<<Реализация алгоритмов цифровой обработки широкополосных сигналов>>}} %%% Название НИР и генерация титульного листа


\Executors %% Список исполнителей здесь
%% это рисует линию размера 3мм и толщиной 0.1 пункт
\begin{longtable}{p{0.35\linewidth}p{0.2\linewidth}p{0.35\linewidth}}
Научный руководитель, 	&		&	\\
доцент Н.~В.~Богач	&\rule{1\linewidth}{0.1pt}	&  \\ \vspace{1cm}

ассистент,  &		&	\\
К.~Д.~Вылегжанина & \rule{1\linewidth}{0.1pt}& \\

студент,  &		&	\\
И.~И.~Иванов & \rule{1\linewidth}{0.1pt}& \\

\end{longtable}


\Referat %% Реферат отчёта, не более 1 страницы
В соответствии с календарным планом проекта \No 123123 настоящий отчёт содержит итоги работ по подэтапу 1 выполнения НИОКР <<Реализация алгоритмов цифровой обработки широкополосных сигналов>>.

На данном этапе проводились работы по определению состава окружения разработки и тестирования, развертывание необходимых средств разработки, создание тестового проекта в среде разработки, подготовка и анализ технической документации по методам цифровой обработки, по техникам программирования под целевую платформу, моделирование алгоритмов цифровой обработки в пакете MatLab.


В результате работы создана модель тракта преобразования сигнала блоками цифровой обработки. Промоделирована последовательности операций цифровой обработки при преобразовании пакета данных, полученных с маяков.

\tableofcontents

%\NormRefs % Нормативные ссылки 

%\Defines % Необходимые определения
 


\Abbreviations %% Список обозначений и сокращений в тексте

\begin{abbreviation}
\item[BPSK] Binary phase shift keying
\item[LTS] Long term support
\item[CI] Continous intergration
\item[ПСП] Псевдослучайная последовательность
\end{abbreviation}

\Introduction
Современные системы цифровой обработки информации широко используют методы расширения спектра, преобразование Уолша-Адамара и помехоустойчивые методы кодирования. 

На данном этапе проводились работы по определению состава окружения разработки и тестирования, развертывание необходимых средств разработки, создание тестового проекта в среде разработки, подготовка и анализ технической документации по методам цифровой обработки, по техникам программирования под целевую платформу, моделирование алгоритмов цифровой обработки в пакете MatLab.

В результате работы создана модель тракта преобразования сигнала блоками цифровой обработки. Промоделирована последовательности операций цифровой обработки при преобразовании пакета данных, полученных с маяков.

\mainmatter %% это включает нумерацию глав и секций в документе ниже
%%#####################################################################
\chapter{Определение состава окружения разработки и тестирования, развертывание необходимых средств разработки}
\section{Окружение тестирования}
\subsection{Ручное, полуавтоматическое тестирование}
\section{Окружение разработки}
\section{Создание тестового проекта в среде разработки}

%%#####################################################################
\section{Техники программирования под целевую платформу}

%%#####################################################################
\chapter{Подготовка и анализ технической документации по методам цифровой обработки}

%%#####################################################################
\chapter{Математическая модель канала цифровой обработки сигналов}

%%#####################################################################

\Conclusion % заключение к отчёту
Задание по реализация алгоритмов цифровой обработки широкополосных сигналов. Проведен полный цикл моделирования в пакете Matlab. Предоставлены исходные коды и сводная информация по проекту.

Все пункты технического задания отражены в отчете.

\begin{thebibliography}{1} %% здесь библиографический список

\bibitem{dspmanual}
{Оппенгейм А., Шафер Р.} 
\newblock {\em Цифровая обработка сигналов}.
\newblock М.: Техносфера, 2006.

\bibitem{filtermanual}
{Гутников В.С.} 
\newblock {\em Фильтрация измерительных сигналов}.
\newblock Л.: Энергоатомиздат, 1990.

\bibitem{codermanual}
{Варгаузин В.А., Цикин И.А.} 
\newblock {\em Методы повышения энергетической и спектральной эффективности цифровой радиосвязи}.
\newblock СПб.: БХВ-Петербург, 2013.

\bibitem{SMS}
{Ипатов В. П., Орлов В. К.}
\newblock {\em Системы мобильной связи}.
\newblock М.: Горячая линия - Телеком, 2003

\bibitem{buch}
{Буч Г.}
\newblock {\em Объектно-ориентированный анализ и проектирование}.
\newblock СПб: Невский диалект, 1999.

\bibitem{badd}
{Бадд Т.}
\newblock {\em Объектно-ориентированное программирование в действии}.
\newblock СПб: Питер, 1997.

\bibitem{Cpp}
{Страуструп Б.}
\newblock {\em Язык программирования C++ Специальное издание}.
\newblock СПб: Бином, 2008.

\bibitem{STL}
{Мейерс С.}
\newblock {\em Эффективное использование STL}.
\newblock СПб: Питер, 2002.

\bibitem{GoF}
{Гамма Э., Хелм Р., Джонсон Р., Влиссидес Дж.}
\newblock {\em Приемы объектно-ориентированного проектирования. Паттерны проектирования}.
\newblock СПб: Питер, 2001.

\bibitem{Qt}
{Шлее М.}
\newblock {\em Qt 4.8. Профессиональное программирование на C++}.
\newblock СПб.: БХВ-Петербург, 2012

\bibitem{agile}
{Вольфсон Б.}
\newblock {\em Гибкие методологии разработки}.

\bibitem{survival}
{Josh Carter}
\newblock {\em New Programmer?s Survival Manual}.
\newblock US Pragmatic Programmers, LLC, 2011

\bibitem{art_r_c}
{Dustin Boswell, Trevor Foucher}
\newblock {\em The art of readable code}.
\newblock US O'Reilly, 2012




\end{thebibliography}

\bibliography{biblio/dsp} %% вместо вставки библиографии можно использовать базы BiBTeX - просто раскомментируйте эту строку.
\newpage 
\chapter{ПРИЛОЖЕНИЕ}
%\section*{Приложение 1. Результаты моделирования в пакете MatLab}
%\lstinputlisting[
%  label={listings:matlabmodel},
%  caption={Исходный код модели тракта},
%  language=Matlab,
%  breaklines=true,
%  numbers=left
%]
%{../code/Simulation/run.m}

\section*{Приложение 2. Результаты работы M-Lint -- утилиты статического анализа кода в MatLab}

\section*{Приложение 3. Результаты работы Dependency Report -- отчет о зависимостях}

\end{document}